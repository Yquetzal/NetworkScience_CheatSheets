\documentclass[addpoints]{exam}
\usepackage[utf8]{inputenc}
\usepackage{xcolor}
\definecolor{light-gray}{gray}{0.95}
\newcommand{\code}[1]{\colorbox{light-gray}{\texttt{#1}}}


\usepackage{tikz,lipsum,lmodern}
\usepackage[most]{tcolorbox}
\usepackage{url}


%This block of commented code translates default words to Spanish
%-------------------------------------------------------------
%\pointpoints{punto}{puntos}
%\bonuspointpoints{punto extra}{puntos extra}

%\totalformat{Pregunta \thequestion: \totalpoints{} puntos}

%\chqword{Pregunta}
%\chpgword{Página}
%\chpword{Puntos}
%\chbpword{Puntos extra}
%\chsword{Puntos obtenidos}
%\chtword{Total}

%\boxedpoints
%-------------------------------------------------------------

\begin{document}
%This code creates the text before the first question
%-------------------------------------------------------------------
\begin{center}
\fbox{\fbox{\parbox{5.5in}{\centering
Experimenting with Gephi}}}
\end{center}

\vspace{5mm}
\begin{tcolorbox}[colback=black!5!white,colframe=white!75!black]
The objective of those exercises is to become familiar with the functionalities of Gephi. Do not hesitate to ask questions if there is something you do not understand \\  
\end{tcolorbox}
\vspace{5mm}

%Here, the questions begin
\begin{questions}

%First question below
\question Visualizing small networks with Gephi
%This question has several parts
\begin{parts}
\part Using Menu \code{File>Open}, open one of the small networks, downloaded from the page of the class.
\part Using the bottom left panel, change the layout. Try in particular \textit{Fruchterman Reingold}, \textit{Yifan Hu}, \textit{expansion}, \textit{noverlap}, \textit{ForceAtlas 2}. Try to play with the parameters of ForceAtlas 2 (\textit{prevent overlap}, \textit{LinLog mode}, etc.).
\part You can move nodes by dragging them. Right clicking on them provide additional functionalities.
\part Zoom in/out using the wheel of your mouse. The position of the cursor is the center of the zoom
\part By clicking with the right button of your mouse on the background and dragging, you can move the window/graph around.
\part Using the top left panel, assign the \textbf{size} of nodes to be proportional to their degree.
\part Using the bottom left panel, change the layout to adapt to these new sizes.
\part Use the button \code{T} at the bottom to display the name of nodes. Using another option at the bottom, make \textbf{node names proportional to node size}.
\part Using the \code{Statistics} tab of the right panel, compute PageRank
\part Using the top left panel, you can now assign a color scale to nodes corresponding to their PageRank score.
\part Have a look at the \code{Data Laboratory} window, accessible by clicking on the button of the same name at the top of your window. Check the data for both Nodes and Edges (panels on the top left)
\part Go back to \code{Overview} window, and, using the right panel, compute the different statistics. Observe the generated plots.
\part Go back to the \code{Data Laboratory} window, and observe that new columns have been created when you computed statistics.
\part Check that you can now change the color and size of nodes (overview, top-left) based on those statistics.
\part Would you say (informally, without comparing with a null model at this point) that the network is a \textit{small world} network ?
\end{parts}
\question Visualizing larger networks with Gephi, and spatial networks
%This question has several parts
\begin{parts}
\part If not done already, using Menu \code{File$>$Open}, open the network called airports from the page of the class.
\part Manipulate it as the previous one
\part From Menu \code{Tools>Plugins}, install the plugin called \code{geolayout}.
\part In the layout panel, you now have a new layout called geolayout. Use the Equirectangular projection and a scale of 1 to position nodes according to a latitude and longitude positions.
\part Use the \code{Filters} tab in the right panel to filter some nodes and/or edges in your graph. For instance, remove the nodes of lower degree,  edges of lower weight, edges of higher betweenness, etc.


\end{parts}

\question Application
%This question has several parts
\begin{parts}
\part Using all what you have learned, create a nice looking visualization of your favorite network.
\part Share your visualization with others in the corresponding Discord channel
\end{parts}
\vspace{5mm}
%\droptotalpoints %Prints the number of points in this question

%The next two questions are multiple choice examples
%-------------------------------------------------------------------




\end{questions}


\end{document}