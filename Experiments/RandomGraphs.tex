\documentclass[addpoints]{exam}
\usepackage[utf8]{inputenc}
\usepackage{xcolor}
\definecolor{light-gray}{gray}{0.95}
\newcommand{\code}[1]{\colorbox{light-gray}{\texttt{#1}}}


\usepackage{tikz,lipsum,lmodern}
\usepackage[most]{tcolorbox}
\usepackage{url}


%This block of commented code translates default words to Spanish
%-------------------------------------------------------------
%\pointpoints{punto}{puntos}
%\bonuspointpoints{punto extra}{puntos extra}

%\totalformat{Pregunta \thequestion: \totalpoints{} puntos}

%\chqword{Pregunta}
%\chpgword{Página}
%\chpword{Puntos}
%\chbpword{Puntos extra}
%\chsword{Puntos obtenidos}
%\chtword{Total}

%\boxedpoints
%-------------------------------------------------------------

\begin{document}
%This code creates the text before the first question
%-------------------------------------------------------------------
\begin{center}
\fbox{\fbox{\parbox{5.5in}{\centering
Experimenting with randomized versions of networks}}}
\end{center}

% \vspace{5mm}
% \begin{tcolorbox}[colback=black!5!white,colframe=white!75!black]
% You can do the exercises in the order that you prefer. \\  
% the \textit{Going further} experiments take more time and I do not expect everyone to do them.
% \end{tcolorbox}
% \vspace{5mm}

%Here, the questions begin
\begin{questions}



\question Comparing networks and their randomized versions.

\begin{tcolorbox}[colback=black!5!white,colframe=white!75!black]
 I recommend to use the airport dataset for interpreting results, but if it is too slow, you can write your code on a smaller network, such as the TV series dataset.
\end{tcolorbox}
%This question has several parts
\begin{parts}
\part Using networkx, load the airport dataset.
\part Generate an ER random version of it. You can use \code{gnp\_random\_graph} or \code{gnm\_random\_graph} methods.  
\part Generate also a configuration model version of it, using \code{expected\_degree\_graph}, and the degrees observed for the real network (e.g., with \code{g.degree})
\part Compare the network properties of the 3 different versions of the graph, at least the average degree, clustering coefficient, average path length. Interpret in terms of small-worldness
\part Compute the (approximate) betweenness, closeness, and PageRank of nodes in the three networks. Compare the largest values between them: are the nodes of highest values the same? Are the highest score similar?
\part Plot the distribution of degrees, betweenness and closeness for each network, and compare them. You can use \code{seaborn} package and functions \code{distplot} and/or \code{scatterplot}+\code{collections.Counter}, and/or \code{ecdfplot}. You can plot with loglog scales (\code{ax.set(xscale="log", yscale="log")}).
\end{parts}    



\vspace{5mm}
\question Going further : Flavors of the Configuration Model

\begin{tcolorbox}[colback=black!5!white,colframe=white!75!black]
We have seen that there are several ways to generate random graphs, usually at least a \textit{soft} and a \textit{sharp} versions (An independent probability to have an edge for each pair of nodes to reach an expected average degree, or a fix number of edges distributed among pairs of nodes). Let's explore the differences in more details.
\end{tcolorbox}
%This question has several parts
\begin{parts}
\part Check the documentation of the \code{configuration\_model}, \code{expected\_degree\_graph} \\ and \code{random\_degree\_sequence\_graph}. Do you see the difference? What are the types returned by these functions, \code{Graph} or \code{MultiGraph}?
\part Try to generate graphs of increasing density. For the sake of simplicity, you can generate uniform distributions, such as the degree raises from 0 to $N-1$.
\part Do you run into problems with some methods?
\part Check the degree distribution of the models, are they identical?
\part Check the number of self-loops and multiple edges. How does it evolves with the density?
\part Generate a power law degree sequence, either manually, or using the \code{random\_powerlaw\_tree\_sequence} method, or extracting it from a real graph. Generate random graphs using the 3 variants.
\part Check using network metrics how different the generated networks are. Most scores are not well defined for MultiGraphs, if you transform it into a simple graph first, how does it affect the results? 
\end{parts}




\end{questions}


\end{document}