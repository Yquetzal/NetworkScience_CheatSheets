\documentclass[addpoints]{exam}
\usepackage[utf8]{inputenc}
\usepackage{xcolor}
\definecolor{light-gray}{gray}{0.95}
\newcommand{\code}[1]{\colorbox{light-gray}{\texttt{#1}}}
\usepackage{listings}
\usepackage{xcolor}

\usepackage{hyperref}
\usepackage{tikz,lipsum,lmodern}
\usepackage[most]{tcolorbox}
\usepackage{url}


%This block of commented code translates default words to Spanish
%-------------------------------------------------------------
%\pointpoints{punto}{puntos}
%\bonuspointpoints{punto extra}{puntos extra}

%\totalformat{Pregunta \thequestion: \totalpoints{} puntos}

%\chqword{Pregunta}
%\chpgword{Página}
%\chpword{Puntos}
%\chbpword{Puntos extra}
%\chsword{Puntos obtenidos}
%\chtword{Total}

%\boxedpoints
%-------------------------------------------------------------

\begin{document}
%This code creates the text before the first question
%-------------------------------------------------------------------
\begin{center}
\fbox{\fbox{\parbox{5.5in}{\centering
Experimenting with Community Structure}}}
\end{center}

% \vspace{5mm}
% \begin{tcolorbox}[colback=black!5!white,colframe=white!75!black]
% You can do the exercises in the order that you prefer. \\  
% the \textit{Going further} experiments take more time and I do not expect everyone to do them.
% \end{tcolorbox}
% \vspace{5mm}

%Here, the questions begin
\begin{questions}



\question Detecting your first Community Structure

\begin{tcolorbox}[colback=black!5!white,colframe=white!75!black]
To detect communities, you can use the \code{cdlib} package. It also contains functions for evaluation and comparison of partitions. For details, check the documentation at \code{\url{https://cdlib.readthedocs.io/en/latest/}}

\textbf{!!!} With google colab, you can install it with \code{!pip install cdlib}, but, due to a colab bug, you also need to do first: 

\code{!pip uninstall python-louvain}
validate with \code{y}, then
\code{!pip install python-louvain}.
\end{tcolorbox}
\begin{parts}
\part Using \code{networkx}, load the  airport dataset, provided as a graphml file. (reminder: you can download it in colab with \code{!wget URL} with URL the url of the file.
\part Using \code{cdlib}, detect communities on this network using the louvain method. You have to use the \code{algorithms.louvain} method (and do \code{from cdlib import algorithms} before.
%\part Add the communities found on your graph as a node property, using \code{NodeCluster.to\_node\_community\_map()} and \code{nx.set\_node\_attributes}. 
\part Visualize the communities found. In order to interpret them, you should draw each node at its geographical location, with a color per community. 

\begin{tcolorbox}[colback=black!5!white,colframe=white!75!black]
There are several ways to draw a spatial network with colors corresponding to communities, from using Gephi to plotting points on an interactive map using \code{folium}.

Here, I provide a simple code to plot the data as a simple scatter plot

\begin{lstlisting}[language=Python, caption=plot on a map]
import seaborn as sns
import matplotlib.pyplot as plt
x= list(nx.get_node_attributes(g,"lon").values())
y= list(nx.get_node_attributes(g,"lat").values())

coms_dict=coms.to_node_community_map()
hues=list(coms_dict[n][0] for n in g.nodes())

plt.figure(figsize=(12,8))
sns.scatterplot(x=x,y=y,hue=hues,palette=sns.color_palette
("tab20",len(coms.communities)),s=5)
\end{lstlisting}
\end{tcolorbox}

\part Vary the resolution parameter and observe changes in the community structure.

\end{parts}    




\vspace{5mm}
\question Comparing Partitions
%This question has several parts
\begin{parts}
\part The provided airport data also contains information about the country of each airport, which can be interpreted as a \textit{ground truth} partition of the network. 

You can obtain it using \code{nx.get\_node\_attributes(g,"country")}. Transform this information into a \code{NodeClustering} object of \code{cdlib}

(\code{nc = NodeClustering(partition,graph,"GroundTruth")}, with \code{partition} a list of list of nodes.
\part Compute the AMI, NMI, or another similarity score (\url{https://cdlib.readthedocs.io/en/latest/reference/evaluation.html}) between the community structure and the partition in countries
\part By exploring with a for loop the values of the resolution parameter for modularity, find the partition with the highest similarity to the partition in countries.
\part Compare visually the results
\end{parts}




\end{questions}


\end{document}